% ----------------------------------------------------------
% Introdução
% ----------------------------------------------------------
\chapter[Introdução]{Introdução}

A rede de computadores do mundo se tornou de longe uma das principais evoluções da humanidade. A globalização é a prova de que todo o avanço funcionou e essa evolução foi necessária. Nesse contexto, a pesquisa sobre segurança da informação foi ampliada, principalmente devido à necessidade.

As vulnerabilidades de sistema entram em contraste com esse acelerado crescimento. Elas mostram como sistemas de grandes empresas podem, facilmente, perecer sob ataques. Os atacantes em sua essência mostram seus poderes no momento do pós ataque, principalmente em redes sociais, quando postam o que fazem, muitas vezes disponibilizando instruções de como fazer.

No contexto da segurança de computadores, vulnerabilidades são falhas ou fraquezas de softwares que permitem a agentes maliciosos subverterem o uso originalmente desejado de algum sistema computacional e violarem políticas de segurança \cite{Belarmino2014}. 

O número de vulnerabilidades que são encontradas em sistemas, nos últimos anos, cresceu de forma exponencial. Isso não significa que todas descobertas foram exploradas. A grande quantidade de demanda para uma baixa quantidade de produção de reparos faz com que apenas vulnerabilidades julgadas como críticas sejam exploradas, uma vez que as restantes são deixadas de lado.

Sistemas inteiros podem ser tomados por hackers, que conseguem ter informações sigilosas sobre os usuários de um serviço, ou mesmo da empresa que mantém tal sistema. Além disso, hackers muitas vezes estorquem empresas a pagarem elevadas quantias para terem seus sistemas e dados privados de volta. 

A discussão sobre as vulnerabilidades em redes sociais mostram como o assunto é tomado pelo público. Assim, pessoas passam a entender mais sobre sistemas e seus bugs, começam a identificar quais são as formas de se identificar vulnerabilidades e quais os passos para apresenta-las ao mundo para que sejam exploradas de maneira legal.

O número de pessoas que discutem temas como vulnerabilidades e bugs em sistemas têm crescido exponencialmente nos últimos anos. O foco é conseguir reduzir o número de falhas em sistemas.

O objetivo desse trabalho é identificar as vulnerabilidades que são discutidas na rede social do twitter, tais como seus aspectos e a possibilidade de ser exploradas. O foco é analisar o volume de dados de discussão sobre esse tópico na rede social, a fim de entender se há um aumento na atenção desse assunto.

Com o crescimento ascendente de novas tecnologias que proporcionam novos sistemas, os backdoors são explorados cada vez mais. A busca pelas vulnerabilidades estão nesse meio para listar quais os possíveis problemas nelas e o que pode ser feito para contorna-las.

Afim de listar as vulnerabilidades discutidas e o volume dessas informações, será criado um sistema que vasculhe o Twitter por publicações sobre o assunto limitando entre uma data x e y. Assim, será possível, através de mineração de dados, quais são as vulnerabilidades mais discutidas, quais os usuários que normalmente fazem tais publicações e o quão retwitadas elas são.

% ------------------%
% --- Objetivos --- %
%\section{Objetivos}


% -------------------%
% --- Resultados --- %
%\section{Resultados esperados}

%\section{Organização do Trabalho}
