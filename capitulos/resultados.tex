% ----------------------------------------------------------
% Resultados
% ----------------------------------------------------------
\chapter[Resultados]{Resultados}

Esse capítulo apresentará os resultados do presente trabalho apresentando gráficos, tabelas e imagens que mostram como foi a resposta obtida do programa criado (Scanner). O capítulo é separado primeiramente pela apresentação dos dados concretizados e como se chegou a eles, posteriormente será apresentado uma comparação com os outros trabalhos. O objetivo é explicar os seguintes pontos:

\begin{itemize}
\item A discussão sobre vulnerabilidades de segurança em redes sociais está aumentando ao longo do tempo assim como a descoberta de nuvas vulnerabilidades?
\item Quais são os tipos de vulnerabilidades com maior insidência de discussão na rede?
\item Qual a razão entre vulnerabilidades de segurança discutidas em redes sociais e vulnerabilidades descobertas?
\item A partir dos resultados das vulnerabilidades encontradas, quais já possuem um exploit?
\end{itemize}

A partir dessas perguntas é possível entender se o tema de vulnerabilidade está ganhando maior interesse por parte de organizações e pela sociedade. Se um dos pilares da humanidade está sendo a tecnologia é preciso que a segurança esteja atrelada a isso, então é necessário que exista pesquisas para o desenvolvimento dessa área.

\section{Análise dos dados}

Para chegar ao objetivo do trabalho foi feito um levantamento da quantidade de vulnerabilidades que foram discutidas no ano de 2019 na rede social do Twitter. Utilizando a ferramenta do Scanner foi feito o filtro nesse período e, a partir dos resultados, foi vinculado à lista do site do CVE para identificação da existência da vulnerabilidade discutida atravé do FindWords. 

Como base de comparação, foi feito também o levantamento para os anos de 2015, 2016, 2017, 2018 e 2019. O filtro utilizado foi o mesmo, palavra 'CVE' para busca, diferenciando-se apenas no intervalo de datas.

%A tabela \ref{tab:tabela1} apresenta a relação entre ano x número de tweets com a palavra chave CVE x número de CVE existentes também no site do NVD.

%\begin{table}[H]
%\centering
%\caption{Índices Sazonais de Descoberta de Vulnerabilidades - Web}
%\label{tab:tabela1}
%\includegraphics[width=.7\textwidth]{imagens/tabela1_Vulnerabilidades.png}
%\end{table}