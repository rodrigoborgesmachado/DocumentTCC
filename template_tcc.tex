% ------------------------------------------------------------------------
% PROPRIEDADES DO DOCUMENTO
% ------------------------------------------------------------------------
\documentclass[12pt,
openright, 
oneside, %
%twoside, %TCC: Se seu texto tem mais de 100 páginas, descomente esta linha e comente a anterior
a4paper,    %
%english,   %
brazil]{facom-ufu-abntex2}

% ------------------------------------------------------------------------
% PACOTES
% ------------------------------------------------------------------------
\pdfstringdefDisableCommands{\let\uppercase\relax}
\usepackage{amsmath}
\usepackage{icomma}
\usepackage[portuguese,ruled]{algorithm2e}
\usepackage{float}
\usepackage{multirow}

% ------------------------------------------------------------------------
% INFO para CAPA e FOLHA DE ROSTO 
% ------------------------------------------------------------------------
\titulo{Análise de conteúdo sobre vulnerabilidades de segurança em redes sociais} %TCC

\autor{Rodrigo Borges Machado} %TCC
\data{2019}

\orientador{Rodrigo Sanches Miani} %TCC
%\coorientador{Nome completo do orientador caso tenha} %TCC

\begin{document}
\frenchspacing 

% ----------------------------------------------------------
% ELEMENTOS PRÉ-TEXTUAIS
% ----------------------------------------------------------
%\pretextual
\imprimircapa
\imprimirfolhaderosto


% --- Inserir folha de aprovação --- %
\begin{folhadeaprovacao}

  \begin{center}
    {\ABNTEXchapterfont\large\imprimirautor}

    \vspace*{\fill}\vspace*{\fill}
    {\ABNTEXchapterfont\bfseries\Large\imprimirtitulo}
    \vspace*{\fill}
    
    \hspace{.45\textwidth}
    \begin{minipage}{.5\textwidth}
        \imprimirpreambulo
    \end{minipage}%
    \vspace*{\fill}
   \end{center}
    
   Trabalho aprovado. \imprimirlocal, 20 de dezembro de 2018: %TCC:

   \assinatura{\textbf{\imprimirorientador} \\ Orientador}  
   \assinatura{\textbf{Maria Adriana Vidigal de Lima} \\Professora}% \\ Convidado 1} %TCC:
   \assinatura{\textbf{Renan Gonçalves Cattelan} \\Professor}% \\ Convidado 2} %TCC:
   %\assinatura{\textbf{Professor} \\ Convidado 3}
   %\assinatura{\textbf{Professor} \\ Convidado 4}
      
   \begin{center}
    \vspace*{0.5cm}
    {\large\imprimirlocal}
    \par
    {\large\imprimirdata}
    \vspace*{1cm}
  \end{center}
  
\end{folhadeaprovacao}
% \includepdf{folhadeaprovacao_final.pdf} % TCC: depois de aprovado o trabalho, descomente esta linha e comente a anterior para incluir o scan da folha de aprovação.

%% OBS.: as seções dedicatória, agradecimento e epígrafe não são obrigatórias. Só as mantenha se achar pertinente.

% --- Dedicatória --- %
\imprimirdedicatoria{Aos meus pais
Hélio Antônio Machado e
Maria Alice Borges da Silveira e \\irmãs Andréa Caroline Machado, Andressa Borges Machado, Luciana Cristina Machado e Simone Cristina Machado}

% --- Agradecimentos --- %
\imprimiragradecimentos{
Agradeço primeiramente a Deus, por possibilitar que eu caminhasse por todo esse caminho de luta com a cabeça sempre erguida, com força para passar pelos obstáculos e saúde para seguir lutando.

Aos meus pais por todo amor, carinho, incentivo e apoio que sempre me deram. Por sempre me incentivarem a buscar mais, a querer mais, e nunca parar de lutar. Por possibilitar uma educação boa e ensinar a ter a humildade que a vida pede.

Às minhas irmãs por todo carinho, apoio e compreensão pelo tempo que tive que abdicar para dedicar a minha formação.

Aos meus amigos pelo incentivo e compreensão durante toda essa jornada. Principalmente por saberem como me animar quando estava em momentos difíceis. Obrigado à Confraria por existir e me fazer tão bem.

À minha namorada Milena Bertolini que sempre esteve ao meu lado enquanto fiz esse trabalho. Obrigado pelo carinho e por todo amor que me deu. Desculpe pelos compromissos cancelados, teremos toda a vida para compensar. Obrigado por estar sempre ao meu lado.

A todos os professores que me insentivaram, desde os professores do meu ensino fundamental, médio e agora no ensino superior. Um obrigado especial aos professores do Instituto Federal do Triângulo Mineiro, que sempre me ajudaram mesmo depois de eu já ter formado no ensino médio.

Ao meu orientador, Professor Doutor Rodrigo Miani, pela paciência, prontidão e confiança. Obrigado por ser sempre um amigo desde as aulas na classe, até no stress desse trabalho. Obrigado por ter sempre um tempo para poder conversar, seja assunto pessoal ou não.
}

% --- Epígrafe --- %
\imprimirepigrafe{"Que todos os nossos esforços estejam sempre focados no desafio à impossibilidade. Todas as grandes conquistas humanas vieram daquilo que parecia impossível."\\Charles Chaplin.}
	
% --- Resumo em português --- %
%
\begin{resumo} %TCC:
Esse trabalho utiliza conceitos de análise de dados para avaliar a ocorrência de vulnerabilidades descobertas em sistemas operacionais Windows, sistemas operacionais diferentes do Windows e aplicações Web nos últimos dez anos usando uma base de dados pública mantida pelo NIST \textit{(National Institute of Standards and Technology)}. Ao longo da pesquisa foi utilizado o cálculo de índice de sazonalidade e a função de autocorrelação para investigar se as tendências encontradas em um estudo anterior continuam as mesmas, além de analisar vulnerabilidades presentes em softwares que não foram contemplados em tal trabalho. Os resultados encontrados indicam duas tendências de sazonalidade: aumento de vulnerabilidades reportadas em junho para todos os sistemas operacionais investigados e diminuição do número de vulnerabilidades em janeiro em todos os softwares analisados.


%% Rodrigo - complemente com uma frase sobre os resultados. Algo do tipo: "Os resultados mostram que..."
 
  \vspace{\onelineskip}
  \noindent
  \textbf{Palavras-chave}: Vulnerabilidade de Segurança, Segurança da Informação, CVE, Twitter, . %TCC:
 \end{resumo}

% --- lista de ilustrações --- %
\listailustracoes

% --- lista de tabelas --- %
\listatabelas

% --- lista de abreviaturas e siglas --- %
\begin{siglas} %TCC:
	
	% 0
	
	% 1
	
	% 2
	
	% 3
	
	% 4
	
	% 5
	
	% 6
	
	% 7
	
	% 8
	
	% 9
	
	% A
    \item[ACF] Análise da Função de Autocorrelação
    \item[AIC] Critério de Informação Akaike
    \item[AML] Modelo Logístico Alhazmi-Malaiya
    \item[AT] Modelo Termodinâmico
    \item[ARIMA] Modelo Auto-Regressivo Integrado de Médias Móveis
	
	% B
	
	% C
    \item[CVE] Commom Vunerabilities and Exposures
	
	% D
	
	% E
	
	% F
	
	% G
	
	% H
    \item[HTTP] Hypertext Transfer Protocol
	
	% I
	\item[IE] Internet Explorer
    \item[IIS] Internet Information Services	
    
    % J
    \item[JRE] Java Runtime Environment
    \item[JW] Modelo Weibull
	
	% K
	
	% L
    \item[LN] Modelo Linear
    \item[LP] Modelo Regressão de Poisson
	
	% M
    \item[SMB] Microsoft Server Message Block
	
	% N
    \item[NIST] National Institute of Standards and Technology
	\item[NVD] National Vulnerability Database
    
	% O
    \item[OS] Sistema Operacional
    \item[OSes] Sistemas Operacionais
    \item[OSVDB] Open Source Vulnerability Database
	
	% P
	
	% Q
	
	% R
    \item[RE] Modelo Exponencial
    \item[RGB] Modelo de Crescimento de Confiabilidade
    \item[RQ] Modelo Quadrático
	
	% S
	\item[SO] Sistema Operacional
    
	% T
    \item[TI] Tecnologia da Informação
	\item[TIC] Tecnologia da Informação e Comunicação
    
	% U
	
	% W
	
	% V
    \item[VDM] Modelo de Descoberta de Vulnerabilidade
	
	% X
	
	% Y
    \item[YF] Modelo Distribuição Normal Dobrada
	
	% Z

\end{siglas}




% --- lista de símbolos --- %
%\begin{simbolos}
	% 0
	
	% 1
	
	% 2
	
	% 3
	
	% 4
	
	% 5
	
	% 6
	
	% 7
	
	% 8
	
	% 9
	
	% A
	
	% B
	
	% C
	
	% D
	
	% E
	\item[$ \in $] Pertence
	
	% F
	\item[$F$] Valor falso do tipo de dado primitivo Booleano
	
	% G
	\item[$ \Gamma $] Letra grega Gama
	
	% H
	
	% I
	
	% J
	
	% K
	
	% L
	\item[$ \Lambda $] Lambda
	
	% M
	
	% N
	
	% O
	
	% P
	
	% Q
	
	% R
	
	% S
	
	% T
	
	% U
		
	% W
	
	% V
	\item[$V$] Valor verdadeiro do tipo de dado primitivo Booleano
	
	% X
	
	% Y
	
	% Z
	\item[$ \zeta $] Letra grega minúscula zeta
	
\end{simbolos} % caso não existe símbolos no trabalho, comente esta linha

% --- sumario --- %
\sumario


% ----------------------------------------------------------
% ELEMENTOS TEXTUAIS
% ----------------------------------------------------------
\textual

% --- Introdução --- %
% ----------------------------------------------------------
% Introdução
% ----------------------------------------------------------
\chapter[Introdução]{Introdução}

A rede de computadores do mundo se tornou de longe uma das principais evoluções da humanidade. A globalização é a prova de que todo o avanço funcionou e essa evolução foi necessária. Nesse contexto, a pesquisa sobre segurança da informação foi ampliada, principalmente devido à necessidade.

Quando se tem grandes sistemas dedicados a manter informações sensíveis, seja de pessoas físicas ou jurídicas, a segurança se vê necessária. O trabalho de assegurar que informações sejam mesmo secretas é o desafio da informática. Há sempre a possibilidade de alguém sem direitos ter acesso a alguma informação, ou mesmo que o dado a ser salvo seja alterado, perdido ou modificado, esses são os riscos de se manter os dados. 

O risco pode ser definido como a probabilidade de que uma situação física com potencial de causar danos possa ocorrer, em qualquer nível, em decorrência da exposição durante um determinado espaço de tempo a uma vulnerabilidade, que por sua vez é definida como uma fraqueza em um sistema, que pode envolver pessoas, processos ou tecnologia que pode ser explorada para se obter acesso a informações \cite{Peotta2006}. 

Se as técnicas para manter a integridade de uma informação já são difícieis no contexto atual, onde inúmeros sistemas conseguem decifrar criptografias, então apresentar formas de acesso diretamente da informação dificultaria ainda mais a manutenção da segurança. Essas formas são as vulnerabilidades de sistemas e serviços, que fazem com que os atacantes tenham muito mais poder de conseguir capturar alguma informação. 

Na existência de uma vulnerabilidade tem-se um risco que decorre do surgimento de uma ameaça que definimos como qualquer circunstância ou evento com o potencial de causar impacto sobre a confidencialidade, integridade ou disponibilidade da informação \cite{Peotta2006}.

As vulnerabilidades de sistema entram em contraste com o acelerado crescimento da informática. Elas mostram como sistemas de grandes empresas podem, facilmente, perecer sob ataques. As principais fontes de dados sobre incidentes de segurança do Brasil são o Centro de Atendimento a Incidentes de Segurança (CAIS), vinculado à Rede Nacional de Pesquisa (RNP) e o Centro de Estudos, Resposta e Tratamento de Incidentes de Segurança no Brasil (CERT.br). Dados dessas instituições mostram que os incidentes de segurança apresentam uma forte tendência de crescimento e cada vez maior impacto. A revolução digital, desenvolvimento das tecnologias e o crescimento das vendas de dispositivos com acesso à Internet são alguns dos fatores que podem explicar este crescimento \cite{Miani2013}.

A Figura \ref{fig:intro1} mostra em resumo o número de vulnerabilidades encontradas no dia 24/10/2019, na semana corrente à essa data, nos meses 10/2019 e 09/2019 e no ano de 2019. 

\begin{figure}[H]
\centering
\includegraphics[width=1\textwidth]{imagens/grafico_cve.png}
\caption{Vulnerabilidades anunciadas recentemente}
\label{fig:intro1}
\end{figure}

Ainda como exemplo, a figura \ref{fig:intro2} apresenta a quantidade de vulnerabilidades já sumarizadas e interpretadas de acordo com sua complexidade o quão críticas são.

\begin{figure}[H]
\centering
\includegraphics[width=1\textwidth]{imagens/grafico_cve_score.png}
\caption{Gráfico que contabiliza um total de vulnerabilidades já interpretadas no ano de 2019}
\label{fig:intro2}
\end{figure}

O número de vulnerabilidades que são encontradas em sistemas, nos últimos anos, cresceu de forma exponencial. Isso não significa que todas descobertas foram exploradas. A grande quantidade de demanda para uma baixa quantidade de produção de reparos faz com que apenas vulnerabilidades julgadas como críticas sejam exploradas, uma vez que as restantes são deixadas de lado.

Vulnerabilidades em softwares têm sido amplamente utilizadas por atacantes, para roubo de informações confidenciais e invasões de redes corporativas. Prover a segurança de um software, porém, não é um objetivo fácil de ser alcançado, dada a complexidade dos sistemas nos dias de hoje \cite{Uto2009}.

Os atacantes que encontram essas vulnerabilidades, em sua essência, mostram seus poderes em redes sociais apresentando seus ataques e compartilhando alguns dados recolhidos. Muitas vezes disponibilizam até instruções de como fazer o ataque. Alguns pedem para outros usuários instalarem aplicativos em suas máquinas para ajudar nos ataques, assim suas máquinas atuam como bootnets.

No contexto da segurança de computadores, vulnerabilidades são falhas ou fraquezas de softwares que permitem a agentes maliciosos subverterem o uso originalmente desejado de algum sistema computacional e violarem políticas de segurança. 

Um software seguro é aquele que satisfaz os requisitos implícitos e explícitos de segurança em condições normais de operação e em situações decorrentes de atividade maliciosa de usuários \cite{Uto2009}. Cada fase de um ciclo de desenvolvimento de software seguro, portanto, tem sua parcela de contribuição para a qualidade do resultado final e não pode ser omitida durante o processo. Na etapa de especificação, requisitos explícitos de segurança devem ser enumerados e requisitos funcionais devem ser avaliados, para verificar se não introduzem uma vulnerabilidade no sistema \cite{Uto2009}.

Sistemas inteiros podem ser tomados por crakers, que conseguem ter informações sigilosas sobre os usuários de um serviço, ou mesmo da empresa que mantém tal sistema. Além disso, crakers muitas vezes estorquem empresas a pagarem elevadas quantias para terem seus sistemas e dados privados de volta. 

A discussão sobre as vulnerabilidades em redes sociais mostram como o assunto é tomado pelo público. Assim, pessoas passam a entender mais sobre sistemas e seus bugs, começam a identificar quais são as formas de se identificar vulnerabilidades e quais os passos para apresenta-las ao mundo para que sejam exploradas de maneira legal.

O número de pessoas que discutem temas como vulnerabilidades e bugs em sistemas têm crescido assim como o número de vulnerabilidades. O foco é conseguir reduzir o número de falhas em sistemas. A discussão do tema leva as pessoas a entenderem mais sobre o assunto, assim mais pessoas contribuem em encontrar falhas em serviços e sistemas.

A exploração de uma vulnerabilidade descoberta em um sistema ou serviço faz surgir uma ameaça. Essa, é tratada já como a concretização de um ataque. Uma ameaça consiste em uma possível ação que, se concretizada, poderá produzir efeitos indesejados ao sistema, comprometendo a confidencialidade, a integridade, a disponibilidade e/ou a autenticidade. Já o ataque é a concretização de uma ameaça, através da exploração de alguma vulnerabilidade do sistema, executado por algum intruso de forma maliciosa ou não \cite{Mello2006}.

Apesar de todo o investimento realizado contra as ameaças à segurança da informação, a quantidade de ataques a empresas e seus aplicativos vem aumentando mais rapidamente do que a nossa capacidade em poder enfrentá-los \cite{ALVESBATISTA2007}.

O objetivo desse trabalho é identificar as vulnerabilidades que são discutidas na rede social do twitter, tais como seus aspectos e a possibilidade de ser exploradas. O foco é analisar o volume de dados de discussão sobre esse tópico na rede social, a fim de entender se há um aumento na atenção desse assunto. Como nos últimos anos houve um notável aumento no volume de vulnerabilidades- espera-se que o aumeto de pessoas interessadas também tenha evoluído. Será criado um sistema que fará uma busca na API GetOldTweets \cite{Pythoncommunity} através de filtros pré selecionados afim de conseguir uma base de dados para verificação dos tweets a partir de tal busca. Assim, com o sistema criado, será feito uma análise dos dados para levantar, quantativa e qualitativamente os dados de segurança discutidos no Twitter tendo em vista as vulnerabilidades já exploradas.

Com a criação desse sistema que consiga fazer uma busca no twiter através de alguns parâmetros, como palavras chaves, filtro de data etc, o objetivo é capturar o resultado em um arquivo e nele fazer uma mineração de dados para identificar palavras recorrentes. Com uma busca de índice invertido, ainda, identificar quais são as palavras recorrentes e combina-las com as vulnerailidaes já conhecidas pelo CVE.

Afim de listar as vulnerabilidades discutidas e o volume dessas informações, será possível com o sistema, através de mineração de dados, identificar quais são as vulnerabilidades mais discutidas, quais os usuários que normalmente fazem tais publicações e o quão retwitadas elas são na rede social.

% ------------------%
% --- Objetivos --- %
%\section{Objetivos}


% -------------------%
% --- Resultados --- %
%\section{Resultados esperados}

%\section{Organização do Trabalho}


% ---  Fundamentação teórica --- %
% ----------------------------------------------------------
% revisão bibliográfica
% ----------------------------------------------------------
\chapter[Revisão Bibliográfica]{Revisão Bibliográfica}

Este capítulo apresenta um levantamento bibliográfico sobre segurança da informação e vulnerabilidades. Tem como objetivo definir conceitos e apresentar exemplos da utilização dos termos e seus exemplos no mundo real.

% --------------------------%
% --- Conceitos Básicos --- %
\section{Segurança da Informação}

Segurança da Informação é o conjunto de orientações, normas, procedimentos, políticas e demais ações que têm por objetivo proteger o recurso informação, possibilitando que o negócio da organização seja realizado e a sua missão seja alcançada
\cite{Fontes2017}.

Proteger a informação é responsabilidade de cada pessoa na organização em que coopera e cabe à organização orientar seus colaboradores em relação à proteção da informação. De acordo com \citeonline{Spanceski2004}, uma da formas de proteger a informação é conhecer os seguintes pilares da área da segurança da informação:

\begin{itemize}
\item Autenticidade: assegura que as informações vieram da fonte anunciada;
\item Confidencialidade: protege as informações contra leituras ou cópias por pessoas não autorizadas;
\item Integridade: garante que as informações não sofram alterações indevidas sem a permissão dos proprietários das informações;
\item Disponibilidade: assegura que as informações estejam acessíveis e utilizáveis aos usuários sempre que necessários.
\end{itemize}

A importância da segurança cresce ainda mais quando os dados da organização são expostos não somente para os colaboradores mas para os usuários ou clientes finais. A falta de um planejamento em segurança pode acarretar problemas em cada um dos pilares descritos acima. Por exemplo, se um banco de dados de senhas não foi devidamente protegido contra leitura por pessoas não autorizadas, um eventual vazamento do mesmo pode representar um grande risco à confidencialidade dos envolvidos. 

\section{Vulnerabilidades}

No contexto da segurança de computadores, uma vulnerabilidade pode ser definida como uma falha em um sistema que permite a realização e a concretização de um ataque em um sistema computacional \cite{Aparecido2014}. Para explorar uma vulnerabilidade, um atacante deve ter pelo menos uma ferramenta ou técnica aplicável que possa conectar a uma fraqueza do sistema, essas ferramentas são chamadas de \textit{exploits} \cite{Whitman2011}. 

De maneira geral, vulnerabilidades possuem um ciclo de vida conforme a Figura \ref{fig:xiao}, onde a partir da descoberta de vulnerabilidades há uma corrida entre desenvolvedores/usuários e atacantes. Enquanto os desenvolvedores tentam liberar atualizações (\textit{patches}) para que os usuários possam instalar e não estarem mais vulneráveis, os atacantes tentam explorar essas vulnerabilidades por meio de ferramentas automatizadas antes que os usuários instalem as correções. 

\begin{figure}[H]
\centering
\includegraphics[width=1\textwidth]{imagens/figura_xiao.PNG}
\caption{Ciclo de vida de uma vulnerabilidade \cite{xiao2018patching}}
\label{fig:xiao}
\end{figure}

Durante a fase de descoberta de vulnerabilidades, quando desenvolvedores ou atacantes descobrem falhas no sistemas, essas vulnerabilidades podem ser divulgadas ao público. Essa divulgação pode ocorrer ou em fóruns públicos ou através da liberação de uma atualização para a correção da vulnerabilidade.

Dados sobre vulnerabilidades de segurança em software geralmente são encontradas usando portais de busca especializados em armazenar e manter informações sobre vulnerabilidades e falhas de segurança, tais como as organizações NVD \cite{NVD}, Secunia \cite{Secunia2009}, US-CERT \cite{CERT1991} e \textit{Open Source Vulnerability Database} \cite{OSVDB2002}. Para este trabalho, assim como para a maioria dos estudos acadêmicos sobre o assunto, a coleta dos dados é feita com o auxílio do portal de busca NVD.

O NVD é uma referência na coleta de dados sobre vulnerabilidade e é sincronizado com o CVE \textit{(Common Vulnerabilities and Exposures)} \cite{CVE1985}. Enquanto o CVE cadastra vulnerabilidades, o NVD categoriza e avalia os riscos delas. 

O CVE é um  banco de dados público em que todos os interessados podem obter acesso a informações sobre vulnerabilidades. O principal gestor do CVE é o MITRE \textit{(Massachusetts Institute of Technology's Digital Computer Laboratory)} e sua proposta não é somente divulgar informações sobre vulnerabilidades de segurança, e sim padronizar essas informações, com a ajuda do NVD \cite{Peotta2006}. 

Um exemplo de vulnerabilidade reportada pelo NVD é mostrada na Figura \ref{fig:nvd1}. A plataforma do NVD disponibiliza todos os CVEs existentes desde 1988 até o ano atual. Para cada ano é possível escolher um mês e visualizar todas as vulnerabilidades divulgadas naquele período, mas não são todos os anos que tem os doze meses divulgados. Foi escolhido para esse exemplo o mês de abril do ano de 2012. 

\begin{figure}[H]
\centering
\includegraphics[width=1\textwidth]{imagens/nvd_exemplo1.png}
\caption{Vulnerabilidades reportadas pelo NVD em abril de 2012}
\label{fig:nvd1}
\end{figure}

A Figura \ref{fig:nvd1} mostra parte das 228 vulnerabilidades divulgadas para o ano de 2012 no mês de abril. Ao selecionar qualquer CVE, uma nova página é carregada com os detalhes da vulnerabilidade escolhida. Para este exemplo, o primeiro CVE-2011-4042 é apresentado na Figura \ref{fig:nvd3}.

\begin{figure}[H]
\centering
\includegraphics[width=1\textwidth]{imagens/nvd_exemplo3.png}
\caption{Vulnerabilidade CVE-2011-4042}
\label{fig:nvd3}
\end{figure}

A Figura \ref{fig:nvd3} mostra alguns detalhes relacionados a vulnerabilidade CVE-2011-4042.  Descrição \textit{(Description na Figura)}, a tabela de impacto \textit{(Impact na Figura)} e a data de publicação da vulnerabilidade \textit{(NVD Published Date na Figura)}, são alguns dos campos apresentados em arquivos XML sendo identificados no cabeçalho principal do arquivo. Esse cabeçalho serve para facilitar na busca dos dados para a coleta.

% -------------------------------%
% --- Trabalhos Relacionados --- %
\section{Trabalhos Relacionados}

Esta seção tem como objetivo descrever trabalhos relacionados ao tema principal desse trabalho: análise de descoberta de vulnerabilidade. Os trabalhos propostos por \citeonline{Massacci2014}, \citeonline{Alhazmi2007}, \citeonline{Alhazmi2008} e \citeonline{Joh2016} serão apresentados a seguir. Ao final, será discutido brevemente a relação destes trabalhos com a proposta deste trabalho.

Um modelo de descoberta de vulnerabilidade pode ser usado para avaliar e prever tendências futuras acerca de descoberta de vulnerabilidade de software. Ou seja, os modelos usam dados antigos de vulnerabilidades divulgadas (em geral fornecidos por bases de dados que catalogam vulnerabilidades de segurança, como a NVD) para tentar prever a ocorrência de novas vulnerabilidades no futuro.

Dentre os diversos modelos estatísticos usados para analisar as características do processo de descoberta de vulnerabilidade, têm-se:

\begin{itemize}
\item Modelo Logístico Alhazmi-Malaiya (AML) e Modelo Linear (LN) proposto por Alhazmi e Malaiya \cite{Alhazmi2006};
\item Modelo Quadrático (RQ) e Modelo Exponencial (RE) proposto por \cite{Rescorla2005};
\item Modelo Weibull (JW) proposto por Kim et al. \cite{Joh2008a};
\item Modelo Distribuição Normal Dobrada (YF) proposto por Younis et al. \cite{Younis2011};
\item Modelo Termodinâmico (AT) proposto por \cite{Anderson2002};
\item Modelo Regressão de Poisson (LP) proposto por Musa e Okumoto \cite{Musa:1984:LPE:800054.801975}.
\end{itemize}

Alguns modelos de crescimento de confiabilidade de software (RGB) também usam VDMs para estudar o processo de detecção de defeitos \cite{Ozment2006}, \cite{Musa2004} em sistemas computacionais.

\citeonline{Massacci2014} apresentam uma metodologia empírica que avalia sistematicamente o desempenho de VDMs em duas dimensões (qualidade e previsibilidade) e abordam todas as questões identificadas da metodologia tradicional, isto é, questões que distorciam estudos prévios no campo. 

Para ilustrar a metodologia, os autores fizeram um experimento para avaliar oito VDMs (AML, AT, LN, JW, LP, RE, RQ e YF) em 30 lançamentos principais de quatro navegadores da Web: IE, Firefox, Chrome e Safari. Classificaram a idade da versão de um navegador em três períodos diferentes: juventude (dentro de 6 - 12 meses desde a data de lançamento), idade média (12 a 36 meses desde a data de liberação) e idade avançada (36 meses). 

Os autores descobriram que se uma versão é relativamente nova, então o melhor modelo a ser usado é um modelo linear para estimar as vulnerabilidades nos próximos 3 - 6 meses. Para navegadores de meia-idade e idade avançada, o melhor é usar um modelo logístico em forma de "s". Em comparação com a metodologia tradicional, isso teria sido impossível.

\citeonline{Alhazmi2007} investigam em seu artigo se é possível prever o número de vulnerabilidades que podem estar potencialmente presentes em um sistema de software, mas que ainda não foram encontradas. Os autores usaram vários sistemas operacionais como representantes de sistemas de software complexos, tais como Windows 95, Windows 98, Windows XP, Windows NT4, Windows 2000, Windows 2003, Red Hat 6.2 e Red Hat 7.1.

Além dos dados sobre vulnerabilidades descobertas nesses sistemas, os autores examinaram os resultados para determinar se a densidade de vulnerabilidades em um programa é uma medida útil tanto para os usuários quanto para os desenvolvedores. A densidade de vulnerabilidades pode ser usada pelos usuários para avaliar o risco ao considerar a compra de um novo sistema de software da mesma empresa e os desenvolvedores podem usar essa métrica para estimar quantidades de vulnerabilidades que provavelmente serão descobertas dentro de algum período de tempo futuro para decidir se adia a aplicação de patches examinando os riscos de modo a evitar a desestabilização dos sistemas de software.

Por fim, os autores consideraram a taxa de descoberta de vulnerabilidades para verificar se os modelos de descoberta de vulnerabilidades podem ser desenvolvidos para projetar tendências futuras. Os resultados revelaram que é possível modelar a descoberta de vulnerabilidades usando um modelo logístico (AML) que às vezes pode ser aproximado por um modelo linear.

\citeonline{Alhazmi2008} descrevem a aplicabilidade e a significância dos parâmetros de seis modelos de descoberta de vulnerabilidade (AT, AML, LM, LP, RQ, RE) para quatro sistemas operacionais (Windows XP, Windows 95, Red Hat Linux 6.2 e Red Hat Fedora). Os modelos de descoberta de vulnerabilidade foram examinados usando o critério de informação de Akaike (AIC) e teste qui-quadrado. A avaliação descobriu que o modelo AML é geralmente melhor a longo prazo, com melhor desempenho para sistemas como o Windows 95, Red Hat Linux 6.2 e Red Hat Fedora. 

Além do trabalho publicado em 2009, os autores Joh e Malayia publicaram um outro trabalho, com uma metodologia semelhante, em 2016 \cite{Joh2016}. Este último trabalho condensa os dados entre 1995 e 2015. Os autores examinaram vulnerabilidades divulgadas em diversos softwares (Sistemas operacionais Windows: Windows NT, Windows 95, Windows 98, Windows 2000, Windows XP e Windows 7; Sistemas operacionais diferentes do Windows: iOS, MAC OS X, Red Hat Linux Enterprise, AIX, Android e Chrome OS; e outros sistemas: Apache, IIS, Internet Explorer, Firefox, Safari e Java (JRE)) para investigar possíveis variações anuais nos processos de descoberta de vulnerabilidades. Também examinaram a periodicidade semanal na distribuição de atualizações de segurança (patches) e exploração das vulnerabilidades. 

Para todos os grupos de software examinados pelos autores, uma taxa de descoberta de vulnerabilidades mais alta é encontrada em determinados meses. Nos produtos Microsoft, uma incidência mais alta durante os períodos de meio ano é observada. Além disso, o comportamento periódico de 7 dias foi observado nos dados de varredura de vulnerabilidades; maior atividade durante a semana  do que nos fins de semana foi confirmada. Especificamente, os valores da atividade de vulnerabilidade correspondentes a terça-feira tenderam a ser maiores do que os outros dias da semana. Os resultados mostraram que a periodicidade precisa ser considerada para a alocação ótima de recursos e para a avaliação dos riscos de segurança.

Apesar dos trabalhos \citeonline{Massacci2014}, \citeonline{Alhazmi2006}, \citeonline{Alhazmi2008}, \citeonline{Joh2009} e \citeonline{Joh2016} conduzirem estudos para modelar o processo de descoberta de vulnerabilidade de software, somente os dois últimos detalham a questão da sazonalidade. O objetivo do presente trabalho é comparar os resultados da análise da sazonalidade em dois diferentes momentos: entre 1995-2007 e entre 2008-2017. A principal motivação para isso é avaliar a evolução do comportamento de questões de segurança ao longo dos anos.

% --- Desenvolvimento --- %
% ----------------------------------------------------------
% Desenvolvimento
% ----------------------------------------------------------
\chapter[Metodologia]{Metodologia}

Neste capítulo será apresentado a metodologia feita no trabalho. Será apresentado e exemplificado o sistema criado no projeto e como foi feito as etapas para criação do mesmo. Será explicado também as tecnologias envolvidas nos sistemas.

\section{Ferramenta Desenvolvida}
Para o desenvolvimento do projeto foi desenvolvido um sistema batizado de Scanner. Esse sistema é quem faz um filtro no Twitter \cite{JackDorseyNoahGlassBizStone2006} nas suas publicações. O scanner é um sistema dividido em 3 sistemas distintos:

\begin{itemize}
\item TwitterSearch: Sistema que fornece um formulário para preenchimento dos filtros a serem utilizados na busca do Twitter.
\item Tweet: Software que faz conexão com a API de comunicação com o Twitter enviando os filtros da busca e interpretando seu retorno em uma base de dados.
\item APIConectorJson: Aplicativo que organiza a base de dados de retorno afim de organizar a informação em um banco de dados, um arquivo xml, um arquivo csv e um arquivo json.
\end{itemize}

Assim, o Scanner funciona de modo que o usuário faça sua busca no Twitter e tenha seu resultado armazenado em alguma base de dados. Então o usuário tem o total acesso dos tweets de resultado da sua busca. Esse acesso é fornecido a partir da API do GetOldTweets \cite{Pythoncommunity}. Será apresentado a seguir a especificação de cada parte do sistema infcluindo a descrição da API utilizada no trabalho.

Para esse trabalho o objetivo é levantar as vulnerabilidades discutidas na rede social Twitter. A partir dos trabalhos similares a esse, foi utilizado filtros já pré selecionados para encontrar essas vulnerabilidades. Com esses resultados, foi feito um estudo dos dados desses tweets, assim como dos usuários que os postam normalmente.

Foi criado também um sistema de busca que tem o objetivo de, a partir dos resultados do Scanner, verificar quais tweets estão realmente relacionados com alguma vulnerabilidade presente no site do NVD. Esse sistema foi batizado de FindWords. Ele será descrito nas seções posteriores.

\subsection{TwitterSearch}
Esse é um sistema cujo objetivo é acionar os demais organizando a informação dos filtros de entrada para a busca dos dados. Desenvolvido em C\#, o sistema apresenta um formulário para preenchimento dos filtros desejados e,  após preenchido o formulário pelo usuário, cria um arquivo JSON que servirá de entrada para os outros sistemas como filtro. O próprio sistema já faz a chamada dos outros aplicativos necessários.

A figura \ref{fig:TwitterSearch} apresenta a primeira parte da tela principal do sistema.

\begin{figure}[H]
\centering
\includegraphics[width=10cm]{imagens/TwitterSearch.png}
\caption{Parte da tela principal do sistema}
\label{fig:TwitterSearch}
\end{figure}

Essa tela apresenta os dados do formulário referente a:

\begin{itemize}
\item Texto e Usuário:

\begin{itemize}
\item Busca de Texto: Essa busca irá filtrar no Twitter publicações que, em seu conteúdo, tenham a palabra digitada. Vale ressaltar que essa palavta será comparada de forma limpa e completa, ou seja, o texto para ser considerado igual deve ter a mesma palavra em seu conteúdo, não apenas parte dela.
\item Usuário: Irá pesquisar na rede social publicações do usuário digitado. Esse usuário é o nome que o mesmo utiliza na rede.
\item Utiliza Top 10 Tweets: Essa opção selecionará os Top 10 tweets do usuário digitado na opção anterior. Esse top 10 significa os 10 tweets mais importante daquele usuário, ou seja, os que faram mais retweetados e curtidos.
\item Usuário: Irá pesquisar na rede social publicações do usuário digitado. Esse usuário é o nome que o mesmo utiliza na rede.
\end{itemize}

\item Intervalo de Datas

\begin{itemize}
\item Utiliza Intervalo de datas: Essa opção ativa o intervalo de datas. Se selecionada, o sistema irá filtrar a data da pubicação do tweet no intervalo das datas selecionadas nos combos.
\item Data Inicial: Data inicial a se considerar os tweets do filtro. A data inicial é incluída no filtro também, ou seja, tweets publicados no dia selecionado também serão retornados.
\item Data Final: Data final a se considerar os tweets do filtro. A data final é incluída no filtro também, ou seja, tweets publicados no dia selecionado também serão retornados.
\end{itemize}

\item Localização

\begin{itemize}
\item Utiliza Local: Essa opção atia o filtro de localização onde o tweet foi publicado. Essa opção não considera a localização do usuário que postou o tweet, mas sim a localização que estava no momento que que foi publicado.
\item Local: Esse filtro pode considerar cidade, estado, país ou mesmo região. Um exemplo de utilização seria: Berlin, Germany.
\item Área: Perímetro a se considerar a partir do local selecionado. Uma opção de entrada seria 50km, então utilizando a opção anterior, de Berlin, esse filtro contataria até 50km em volta dessa cidade. 

\item Utiliza Geolocalização: Essa opção permite que o usuário utiliza geolocalização para filtrar os tweets. assim como o Local, a geolocalização irá identificar a partir do local em que o tweet foi publicado.
\item Geolocalização: Entrada de latitude e longitude. Um de entrada exemplo seria: 55.75, 37.61. A Área, já mencionada, também funciona se preenchida assim como a geolocalização.
\end{itemize}

\end{itemize}

Essa é a primeira tela do sistema. Esses filtros colocados são os principais e comumente mais utilizados. O trabalho \citeonline{Sabottke2015} utiliza filtro de texto e data para obtenção dos resultados. Em sua pesquisa, os dados têm por base o ano de 2015 filtrando a palabra CVE, que é a sigla utilizada para filtro de vulnerabilidades \cite{CVE1985}.

A figura \ref{fig:TwitterSearch2} apresenta a segunda parte da tela principal do sistema. Como o número de filtros é grande, o sistema acaba tendo uma barra de rolagem que disponibiliza o restante dos filtros.

\begin{figure}[H]
\centering
\includegraphics[width=10cm]{imagens/TwitterSearch2.png}
\caption{Segunda Parte da tela principal do sistema}
\label{fig:TwitterSearch2}
\end{figure}

A sequnência da tela apresenta os dados do formulário referente a:

\begin{itemize}

\item Limite de Tweets

\begin{itemize}
\item Utiliza Limite de Tweets: Essa opção valida se vai limitar o número de tweets no retorno da bucsa.
\item Limite: Número de Tweets máximo de retorno da consulta
\end{itemize}

\item Saída

\begin{itemize}
\item Tipos de Saída: Essa opção seleciona qual o tipo de saída do programa: JSON, CSV, XML, SQL.
\item Gerar Logs Detalhados: Essa opção faz com que os sistemas gerem um log detalhado sobre suas atividades.
\item Caminho de Saída: Essa opção permite que o usuário selecione onde os arquivos de saída devem ser copiados.
\end{itemize}

\end{itemize}

Esse sistema é a primeira parte do trabalho, cujo objetivo é apenas apresentar o formulário e, a partir do preenchimento dele, gerar um arquivo JSON de configuração de filtro de entrada para os outros arquivos.
 
\subsection{Tweet}

Esse sistema foi desenvolvido em Python e tem o objetivo de fazer a conexão com a API do GetOldTweets \cite{Pythoncommunity}. Para sua execução, ele precisa de um arquivo de entrada que contenha as informações dos filtros que será colocado na busca dos tweets. A imagem \ref{fig:entradaTweet} mostra um exemplo de um arquivo de entrada desse sistema.

\begin{figure}[H]
\centering
\includegraphics[width=8cm]{imagens/entradaTweet.png}
\caption{Exemplo de entrada do programa Tweet}
\label{fig:entradaTweet}
\end{figure}

Esses filtros são os enviados ao GetOldTweets\cite{Pythoncommunity} para obter o retorno. Desse retorno, para cada tweet x de resultado, é buscado o TOP 10 dos tweets referente àquele usuário que publicou o tweet x.

O sistema conta ainda com a opção de log, na figura \ref{fig:entradaTweet} apresentado na tag "log". Essa opção faz com que cada passou do sistema seja relatado em um arquivo de log no caminho: raiz da aplicação/log/"data do log". Essa opção é importante, pois nem sempre a API está funcionando corretamente, o que faz com que o sistema apresente lentidão em sua execução. 

Ademais, o sistema faz uma interação com o usuário afim de apresentar em qual parte da análise está sendo efetuada, conforme imagem \ref{fig:TweetExecucao}.

\begin{figure}[H]
\centering
\includegraphics[width=8cm]{imagens/TweetExecucao.png}
\caption{Exemplo de execução do sistema Tweet}
\label{fig:TweetExecucao}
\end{figure}

O sistema apresenta a data e hora de início da execução. Após isso ele começa a realizar a requisição da API, quando finalizado apresenta a data e hora de finalização requisição. Após isso o Tweet chama a aplicação APIConectorJson, que faz o tratamento dos dados, quando finalizado apresenta data e hora da finalização do taratamento dos dados. Apresenta ainda a saída da requisição: 0 para sucesso e qualquer outro para código de erro do sistema operacional. No final apresenta o caminho onde os arquivos foram salvos.

O sistema funciona então em duas fases: (1) seleciona os filtros no arquivo de entrada e faz a requisição com a API gerando um arquivo de saída X e (2) seleciona esse arquivo de saída e chama o aplicativo APIConectorJson passando o arquivo X como entrada e passando quais são os tipos de saída desejados. 

\subsection{API GetOldTweets}

A API do GetOldTweets é uma biblioteca desenvolvida em python que tem o objetivo de fazer uma busca no Twitter apartir dos filtros possíveis, que serão descritos posteriormente. Seu funcionamento se baseia na leitura de um brownser que acessa a rede social, então o resultado é obtido a partir dessa busca. Logo, o resultado sempre fica ordenado de forma cronológica, ou seja, pela data da publicação.

Essa API possui vários recursos para ser utilizado, não se limitando apenas à uma bilioteca. É possível acessa-la através de linha de comando pelo CMD passando os parâmetros de busca, assim como é possível chama-la a partir de um programa externo, seja em python ou não. Cada uma das formas de chama-la gera um resultado diferente: se chamado por linha de comando é gerado um arquivo CSV com as informações, já chamando a partir de um programa externo a bilioteca retorna um arquivo JSON com os tweets. A estrutura dos objetos de retorno é dada com os seguintes atributos:

\begin{itemize}
\item id (string); Código referente ao tweet
\item permalink (string); Link da publicação
\item username (string); Nome do usuário que fez a publicação
\item text (string); Texto do tweet
\item date (datetime); Data da publicação
\item retweets (integer); Número de vezes que o tweet foi retweetado
\item favorites (integer); Número de vezes que o tweet foi favoritado
\item mentions (string); Lista as menções da publicação
\item hashtags (string); Apresneta as hashtags da publicação
\item geo (string); Localização onde o tweet foi publicado
\end{itemize}

Para o presente trabalho foi escolhido utilizar a API como uma biblioteca de um sistema externo, então é conectado na API e retornado o arquivo JSON com os dados da busca. 

A biblioteca funciona em duas partes: listar os filtros para a busca (classe TwitterCriteria) e realizar a busca propriamente dita (TweetManager). A classe de organização das buscas possui os seguintes métodos que possibilitam a realização dos filtros:

\begin{itemize}
\item setUsername (str or iterable): Nome do usuário para a busca
\item setSince (str. "yyyy-mm-dd"): Data de início da busca. Se não for dado um valor não há limite de busca. Se setado um valor esse é incluído no resultado, ou seja, se colocado a data 2015-01-01 no resultado haverá tweets dessa data.
\item setUntil (str. "yyyy-mm-dd"): Data final da busca. Se não colocado a data final será o dia em que a busca foi realizada, com resultados incluindo esse dia. Se setado o dia colocado não é considerado no resultado, ou seja, se colocado 2015-12-31 não haverá tweets dessa data.
\item setQuerySearch (str): Texto para verificação no Twitter. Esse texto será buscado na rede como um todo e deve haver um "match" completo da palavra para o tweets ser considerado resultado da busca. Ou seja, se filtrado a palavra "CVE", como será feito por esse trabalho, será retornado todos os tweets da rede que tenham essa palavra. A busca não é case sensitive, ou seja, não importa texto em maiúsculo e minúsculo.
\item setTopTweets (bool): Se setado como "true" é retornado apenas os Top tweets. Esse conceito de Top tweet se baseio no número de curtida e número de seguidores do usuário que publicou o tweet.
\item setNear(str): Uma referência de área localização onde o tweet foi publicado. 
\item setWithin (str): Filtra a busca por uma área de localização a partir da opção colocada no setNear.
\item setMaxTweets (int): O número máximo de tweets a ser suportado na busca dos dados. Se não setado ou se o valor for menor que 1 será retornado a busca total dos tweets, sem limite.
\end{itemize}

A biblioteca possui ainda algumas limitações baseada nas condições em que a busca é efetuada. Basicamente, se a rede for oscilante no momento de leitura e requisição dos dados é possível que o retorno não seja composto por todos os tweets da busca, uma vez que a API faz uma busca na rede social lendo um brownser, ou seja, se não carregar os dados no navegador a biblioteca não terá nada para ler, ou se carregar apenas parte irá retornar apenas esse trecho de resultado. Para facilitar a busca de ocorrência desse problema basta conferir no resultado a data do último tweet retornado, se a data for uma não reconhecida, ou seja, não for a desejada, houve erro na busca. Esse erro também pode ocorrer devido ao estouro da memória principal, pois a API faz a busca utilizando dessa memória para armazenamento dos dados para apenas depois montar o arquivo de saída.

Para o presente trabalho foi enfrentado ambos os problemas. Como será citado nos resultado, a busca foi feita para o ano de 2019, mas comparando aos resultados dos anos anteriores (2018, 2017, 2016 e 2015), e para a busca de todo esse período foi encontrado certa dificuldade por limitação de rede e de memória.

\subsection{APIConectorJson}

Essa parte do sistema funciona com o propósito de organizar as informações obtidas da API GetOldTweets \cite{Pythoncommunity}. Ele recebe de entrada um arquivo JSON com os dados da requisição e recebe também quais são as saídas esperadas conforme citado na seção anterior. Desse arquivo é retirado todos os objetos e para cada um dos tipos de saída é feito um tratamento diferente. Além disso, todo o processo também pode ser logado caso haja alguma espécie de erro. Esse log é o mesmo do sistema Tweet.

Os tipos de saída são configurados da seguinte forma:

\begin{itemize}
\item JSON: Arquivo com estrutura pré estabelecida, é um modelo para armazenamento e transmissão de informações no formato texto e que é bastante utilizado por aplicações Web que trabalham com a tecnologia AJAX. É um objeto JavaScript. A própria API do GetOldTweets \cite{Pythoncommunity} retorna o arquivo JSON com o resultado, então o sistema só copia o arquivo para a pasta de saída.  
\item CSV: Arquivo com a estrutura de informações separados por vírgulas. Comunmente utilizado para apresentação de informações em planílhas e cópias de dados entre bases de informações, como banco de dados. Esse tipo de armazenamento agrupa as informações de texto em planilhas. O sistema o utiliza colocando os tipos na primeira linha do arquivo e as informações nas demais.
\item XML: é um tipo de armazenamento que utiliza a linguagem de marcação recomendado pela W3C. É utilizado para representação de dados de forma estruturada mantendo as informações de acordo com suas especificidades. Utiliza de tags para caracterizar uma informação. O sistema utiliza as tags colocando os tipos dos dados nelas e a informação no espaço restante.
\item SQL: é um armazenamento em banco de dados comum, tabelado. A informação é colocada em uma tabela onde a coluna é o tipo de informação e as linhas são as tuplas de informações. O sistema utiliza esse tipo colocando os tipos dos tweets como as colunas da tabela RETORNO e as linhas são os tweets que estavam no arquivo JSON de entrada.
\end{itemize}

O sistema então tem como objetivo organizar os dados do arquivo de entrada nos aruqivos de saída. Ele trabalha apenas com essa função, possibilitando que qualquer arquivo de entrada JSON tenha sua saída nos tipos permitidos de dados: JSON, CSV, XML e SQL. 

\subsection{Scanner}

O conjunto dos sistemas anteriormente citados dá origem ao sistema Scanner. Esse possui a interface do TwitterSearch, o que possibilita preenchimento do formulário disponibilizando um arquivo JSON de configuração de filtragem e que chama o "cérebro" do programa: o Tweet. Este processa o arquivo de entrada com os filtros requisitando a API GetOldTweets \cite{Pythoncommunity} com esses dados, obtendo os tweets de retorno que satisfazem a busca, criando um novo arquivo json de saída contendo uma lista com os objetos dos tweets retornados pela API. Assim chama o APIConectorJson para organização dos dados passando o JSON com os objetos dos tweets e passando os tipos de saída selecionado no TwitterSearch, depois de finalizado o próprio Tweet copia as saídas para o caminho selecionado pelo usuário.

Um exemplo de funcionalidade segue na figura \ref{fig:entrada1} que apresenta as seguintes entradas.

\begin{figure}[H]
\centering
\includegraphics[width=10cm]{imagens/entrada1.png}
\caption{Entrada de exemplo (imagem editada com o propósito de listar apenas os filtros utilizados)}
\label{fig:entrada1}
\end{figure}

Com essas estradas, o sistema chama as outras aplicações e obtem as saídas nos arquivo em JSON, CSV, XML e SQL. Todos esses arquivos possuem os dados de retorno do Twitter com a busca feita. Tomando como exemplo um desses tweets do retorno da busca, o de ID 1190312935895158789 do usuário Tribe\_Secure, ele é listadas nas figuras \ref{fig:saidaJson}, \ref{fig:saidaCSV}, \ref{fig:saidaXML} e \ref{fig:saidaSQL}.

\begin{figure}[H]
\centering
\includegraphics[width=1\textwidth]{imagens/saidaJson.png}
\caption{Exemplo do tweet em saída JSON (o texto foi alinhado para uma melhor visualização)}
\label{fig:saidaJson}
\end{figure}

\begin{figure}[H]
\centering
\includegraphics[width=1\textwidth]{imagens/saidaCSV.png}
\caption{Exemplo do tweet em saída CSV}
\label{fig:saidaCSV}
\end{figure}

\begin{figure}[H]
\centering
\includegraphics[width=1\textwidth]{imagens/saidaXML.png}
\caption{Exemplo do tweet em saída XML}
\label{fig:saidaXML}
\end{figure}

\begin{figure}[H]
\centering
\includegraphics[width=1\textwidth]{imagens/saidaSQL.png}
\caption{Exemplo do tweet em saída SQL}
\label{fig:saidaSQL}
\end{figure}

O exemplo de entrada ainda será utilizado nesse trabalho, filtrando ainda por data. Como o retorno monta uma base de dados consideravelmente grande, ele já mostra que o volume de dados de quando não se coloca limite de número de tweets pode crescer ainda mais. A fim de teste, foi colocado uma busca com um número ilimitado de tweets e sem fechar nenhuma data, como resultado passaram-se mais de 24 horas e o programa ainda não tinha finalizado suas solicitações.

Esse é o sistema no qual o trabalho se baseia. É com ele que a análise sobre as quantidade de vulnerabilidades discutidas no Twitter será possível, assim como a frenquência da mesma nos últimos anos. Esse é um programa genérico que possibilita buscas em toda a rede, logo será nossos filtros que farão com que a busca seja orientada aos objetivos do trabalho.

\subsection{FindWords}

Esse sistema tem como objetivo buscar uma palavra (ou parte dela) em arquivos de diretórios distintos. Dado um diretório A e a palavra "CVE", o programa irá listar como resultado todos os arquivos da pasta e subpasta do diretório A que conténham essa palavra e em qual linha do arquivo a mesma está. 

O objetivo desse programa é buscar para cada resultado do Scanner o código CVE publicado nas listas presente no site NVD. A partir desse resultado é possível verificar quais vulnerabilidades estão sendo discutidas e realmente existem no cite do CVE e do NVD.

O programa possibilita ler todos os possíveis arquivos de saída do Scanner (XML, JSON, CSV e SQL) e, a partir desses arquivos, ler o código descrito na publicação no Twitter e buscar esse código nas bases de conhecimento do NVD.

O resultado dessa busca é um arquivo txt de saída que lista o código CVE encontrado no tweet e o diretório do arquivo e linha do arquivo em que o mesmo código foi encontrado.

Há ainda a possibilidade de, ao invés de procurar nos arquivos do diretório selecionado, utilizar um banco de dados para a busca da palavras.

\section{Coleta de dados}

O Scanner faz uma busca no Twitter a partir da API GetOldTweets \cite{Pythoncommunity} passando os parâmetros de filtro. Essa é a coleta de dados que o sistema faz da rede social. 

Para trabalhar com os dados, eles são organizados em outros arquivos que possibilitam a visualização melhor dos dados, assim o trabalho de análise fica mais simples.

O site do NVD, como já foi citado, apresenta uma lista de opção para download das vulnerabilidades. O site organiza essas informações colocando-as em um arquivo ZIP para cada ano \cite{NISTDownload}. Assim, é possível baixar essas informações e utilizar delas para analisar os dados de retorno. O site do CVE também fornece a opção de baixar o um arquivo XML com os mesmos dados, mas o arquivo não é em formato ZIP e acaba sendo um pouco maior para download \cite{TheMITRECorporation2018}.

A coleta de dados então é feita a partir de duas bases para a análise desse trabalho: Twiiter, através da API do GetOldTweets \cite{Pythoncommunity}; e o NVD, que processa seus dados a partir da base do CVE.

Tomando como exemplo a figura \ref{fig:exemploVulnerabilidade}, tem-se a vulnerabilidade de código CVE-2019-2110. A figura \ref{fig:exemploVulnerabilidadeCVE} apresenta suas informações no site do CVE.

\begin{figure}[H]
\centering
\includegraphics[width=1\textwidth]{imagens/exemploVulnerabilidade.png}
\caption{Exemplo de publicação de uma vulnerabilidade no Twitter em formato xml}
\label{fig:exemploVulnerabilidade}
\end{figure}

\begin{figure}[H]
\centering
\includegraphics[width=1\textwidth]{imagens/exemploVulnerabilidadeCVE.png}
\caption{Descrição da vulnerabilidade CVE-2019-2110 no site do CVE \cite{CVE1985}}
\label{fig:exemploVulnerabilidadeCVE}
\end{figure}

O arquivo disponibilizado no site do NVD referente ao ano de 2019 descreve a vulnerabilidade, como apresentado na figura \ref{fig:descricaoVulnerabilidadeNVD}. Assim como o arquivo XML baixado no site do CVE, apresentado na figura \ref{fig:descricaoVulnerabilidadeCVE}.

\begin{figure}[H]
\centering
\includegraphics[width=1\textwidth]{imagens/descricaoVulnerabilidadeNVD.png}
\caption{Descrição da vulnerabilidade CVE-2019-2110 no arquivo disponibilizado no NVD}
\label{fig:descricaoVulnerabilidadeNVD}
\end{figure}

\begin{figure}[H]
\centering
\includegraphics[width=1\textwidth]{imagens/descricaoVulnerabilidadeCVE.png}
\caption{Descrição da vulnerabilidade CVE-2019-2110 no arquivo disponibilizado no CVE}
\label{fig:descricaoVulnerabilidadeCVE}
\end{figure}

A coleta dos dados então não se restringe apenas ao Twitter, mesmo que o objetivo seja a verificação de discussão sobre vulnerabilidades na rede social. A coleta é feita nos sites que tratam de vulnerabilidades para a identificação correta do que é ou não vulnerabilidade. Parte da análise está em filtrar na rede social, a outra está na identificação dessas vulnerabilidades.

Para o presente trabalho os filtros foram feito para esse ano (o objetivo é identificar como a discussão está hoje!). Logo, tanto no site do NVD quanto no CVE foi baixado os arquivos referente ao ano de 2019 e os mesmos serão utilizados em conjunto com o filtro feito de publicações do ano de 2019 por parte do Scanner.

Então o Scanner é o responsável por buscar esses dados do Twitter e o FindWords será o responsável por relacionar os dados recolhidos da rede social com os dados presentes nos sites do CVE e do NVD.

No próximo capítulo será listado os resultados dessas buscas comparando outros anos. O número de vulnerabilidades tem crescido exponencialmente, isso é fato, o CVE prova isso a partir dos tamanhos dos arquivos de cada ano. Esse aumento também é experado no número de pessoas discutindo o tema no Twitter. 



% --- Conclusão --- %
% ----------------------------------------------------------
% Resultados
% ----------------------------------------------------------
\chapter[Resultados]{Resultados}

Esse capítulo apresentará os resultados do presente trabalho apresentando gráficos, tabelas e imagens que mostram como foi a resposta obtida do programa criado (Scanner). O capítulo é separado primeiramente pela apresentação dos dados concretizados e como se chegou a eles, posteriormente será apresentado uma comparação com os outros trabalhos. O objetivo é explicar os seguintes pontos:

\begin{itemize}
\item A discussão sobre vulnerabilidades de segurança em redes sociais está aumentando ao longo do tempo assim como a descoberta de nuvas vulnerabilidades?
\item Quais são os tipos de vulnerabilidades com maior insidência de discussão na rede?
\item Qual a razão entre vulnerabilidades de segurança discutidas em redes sociais e vulnerabilidades descobertas?
\item A partir dos resultados das vulnerabilidades encontradas, quais já possuem um exploit?
\end{itemize}

A partir dessas perguntas é possível entender se o tema de vulnerabilidade está ganhando maior interesse por parte de organizações e pela sociedade. Se um dos pilares da humanidade está sendo a tecnologia é preciso que a segurança esteja atrelada a isso, então é necessário que exista pesquisas para o desenvolvimento dessa área.

\section{Análise dos dados}

Para chegar ao objetivo do trabalho foi feito um levantamento da quantidade de vulnerabilidades que foram discutidas no ano de 2019 na rede social do Twitter. Utilizando a ferramenta do Scanner foi feito o filtro nesse período e, a partir dos resultados, foi vinculado à lista do site do CVE para identificação da existência da vulnerabilidade discutida atravé do FindWords. 

Como base de comparação, foi feito também o levantamento para os anos de 2015, 2016, 2017, 2018 e 2019. O filtro utilizado foi o mesmo, palavra 'CVE' para busca, diferenciando-se apenas no intervalo de datas.

%A tabela \ref{tab:tabela1} apresenta a relação entre ano x número de tweets com a palavra chave CVE x número de CVE existentes também no site do NVD.

%\begin{table}[H]
%\centering
%\caption{Índices Sazonais de Descoberta de Vulnerabilidades - Web}
%\label{tab:tabela1}
%\includegraphics[width=.7\textwidth]{imagens/tabela1_Vulnerabilidades.png}
%\end{table}

% --- Conclusão --- %
\include{capitulos/conclusao}

% ----------------------------------------------------------
% ELEMENTOS PÓS-TEXTUAIS
% ----------------------------------------------------------
\postextual

% --- Referências bibliográficas --- %
\bibliography{bibliografia}

% --- Apêndices --- %
% só mantenha se for pertinente.
%\begin{apendicesenv}
%\partapendices % Imprime uma página indicando o início dos apêndices

% --- Apendice 1--- %
%\include{apendices/apendice1}

%\end{apendicesenv}

% --- Anexos --- %
% so mantenha se pertinente.
%\begin{anexosenv}
%\partanexos % Imprime uma página indicando o início dos anexos

% --- Anexo 1 --- %
%\include{anexos/anexo1}

%\end{anexosenv}


% ------------------------------------------------------------------------
% FIM DO DOCUMENTO
% ------------------------------------------------------------------------
\printindex
\end{document}